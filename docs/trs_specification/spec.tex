\documentclass{article}

\usepackage[T1]{fontenc}
\usepackage{amsmath, amsthm}
\usepackage{amssymb}
\usepackage{listings}
\usepackage[svgnames]{xcolor}
\usepackage{tikz}
\usepackage{array}
\usepackage{graphicx}
\usepackage[backend=biber, maxbibnames=9]{biblatex}
\usepackage[noend]{algpseudocode}
\usepackage{algorithm}
\usepackage{mathtools}
\usepackage{xcolor}
\usepackage{subcaption}
\usepackage{wrapfig}
\usepackage{hhline}
\usepackage{bsymb}
\usepackage{hyperref}

% Commands
% Simple set comprehension notation
\newcommand{\Set}[2]{%
  \{\, #1 \mid #2 \, \}%
}
% Event-B-like set notation
\newcommand{\bSet}[3]{%
  \{\, #1 \cdot #2 \mid #3 \, \}%
}
\newcommand{\bSetT}[2]{%
  \{\, #1 \cdot #2 \,\}%
}

% Bag notation
\newcommand{\lbbar}{\{\kern-0.5ex|}
\newcommand{\rbbar}{|\kern-0.5ex\}}
\newcommand{\bag}[3]{%
  \lbbar \, #1 \cdot #2 \mid #3 \, \rbbar%
}
\newcommand{\bagT}[2]{%
  \lbbar \, #1 \cdot #2 \,\rbbar%
}

% List comprehension notation
\newcommand{\List}[3]{%
  [\, #1 \cdot #2 \mid #3 \, ]%
}
\newcommand{\ListT}[2]{%
  [\, #1 \cdot #2 \, ]%
}
% List concatenation
\newcommand{\concat}{%
  \mathbin{{+}\mspace{-8mu}{+}}%
}

% From https://tex.stackexchange.com/questions/82782/footnote-in-align-environment
\makeatletter
\let\original@footnote\footnote
\newcommand{\align@footnote}[1]{%
  \ifmeasuring@
    \chardef\@tempfn=\value{footnote}%
    \footnotemark
    \setcounter{footnote}{\@tempfn}%
  \else
    \iffirstchoice@
      \original@footnote{#1}%
    \fi
  \fi}
\pretocmd{\start@align}{\let\footnote\align@footnote}{}{}
\makeatother

% Other preamble
\allowdisplaybreaks
\graphicspath{{./images/}}
% NOTE: need to run `biber <name>` to ensure up-to-date references
\addbibresource{spec.bib}

\title{Complete TRS Specification for Abstract Collection Types}
\author{Anthony Hunt}

\begin{document}
\maketitle
\tableofcontents
\newpage

\section{Introduction}
This document serves as a living specification of the underlying term rewriting system used in the compiler for a modelling-focused programming language.

\section{High Level Strategy}

\paragraph{General Strategy}
A basic strategy to optimize set and relational expressions is:
\begin{enumerate}
  \item Normalize the expression as a set comprehensions
  \item Simplify and reorganize conjuncts of the set comprehension body
\end{enumerate}

\paragraph{Intuition}
The TRS for this language primarily involves lowering collection data type expressions into pointwise boolean quantifications. Breaking down each operation into set builder notation enables a few key actions:
\begin{itemize}
  \item Quantifications over sets ($\bSet{x}{G}{P}$) are naturally separated into generators ($G$) and (non-generating) predicates ($P$). For sets, at least one membership operator per top-level conjunction in $G$ will serve as a concrete element generator in generated code. Then, top level disjunctions will select one membership operation to act as a generator, relegating all others to the predicate level. For example, if the rewrite system observes an intersection of the form $\bSetT{x}{x \in S \land x \in T}$, the set construction operation must iterate over at least one of $S$ and $T$. Then, the other will act as a condition to check every iteration (becoming $\bSet{x}{x \in S}{x \in T}$).
  \item By definition of generators in quantification notation, operations in $G$ must be statements of the form $x \in S$, where $x$ is used in the ``element'' portion of the set construction. Statements like $x \notin T$ or checking a property $p(x)$ must act like conditions since they do not produce any iterable elements.
  \item Any boolean expression for conditions may be rewritten as a combination of $\lnot, \lor$, and $\land$ expressions. Therefore, by converting all set notation down into boolean notation and then generating code based on set constructor booleans, we can accommodate any form of predicate function.
\end{itemize}


\paragraph{Granular Strategy (Sets)}
% TODO take from report.tex, give supporting equations too (all the rules must be stable first)
\begin{description}
  \item[Phase 1: Set Comprehension Construction] Break down all qualifying sets into comprehension forms, collapsing and simplifying where needed.
  \item[Phase 2: DNF Predicates] Revise comprehension predicates to top-level disjunctive normal form. Each or-clause should have at least one feasible generator. Each clause should record a list of candidate generators
  \item[Phase 3: Predicate Simplification] Remove superfluous dummy variables, group or-clauses that use the exact same generator (ex. $\bSetT{x}{x \in S \land x \neq 0 \lor x \in S \land x = 0} \rightarrow \bSetT{x}{x \in S \land (x \neq 0 \lor x = 0)}$). Clauses should be group-able based on DNF, and generators should be selected and recorded.
  \item[Phase 4: Set Code Generation] Converts quantifiers into for-loops and if-statements.
\end{description}

\section{Supported Operations}
\begin{table}[H]
    \centering
    \caption{Summary table: a few operators on sets and relations.}
    \begin{tabular}{|c|c||c|c|}
    \hhline{|--||--|}
    \multicolumn{2}{|c||}{Sets} & \multicolumn{2}{|c|}{Relations} \\
    \hhline{:==::==:}
    Syntax & Label/Description & Syntax & Label/Description\\
    \hhline{|--||--|}
    $set(T)$ & Unordered, unique collection             & $S \pfun T$ & Partial function \\
    $S \leftrightarrow T$ & Relation, $set(S\times T)$  & $S \tinj T$& Total injection\\
    $\emptyset$ & Empty set                             & $a \mapsto b$ & Pair (relational element) \\
    $\{a,b,...\}$ & Set enumeration                     & $dom(S)$ & Domain\\
    $\bSet{x}{x \in S}{P}$ & Set comprehension          & $ran(S)$ & Range\\
    $S \cup T$ & Union                                  & $R[S]$ & Relational image\\
    $S \cap T$ & Intersection                           & $R \ovl Q$ & Relational overriding\\
    $S \setminus T$ & Difference                        & $R \circ Q$ & Relational composition\\
    $S \times T$ & Cartesian Product                    & $S \triangleleft R$ & Domain restriction\\
    $S \subseteq T$ & Subset                            & $R^{-1}$ & Relational inverse\\
    \hhline{|--||--|}
    % $f(S)$ & Function application\\ % is this like function mapping over a set? do we need to include this?
    \end{tabular}
    \label{tab:ADTOps}
\end{table}
\begin{table}[H]
    \centering
    \caption{Collection of operators on set data types.}
    \begin{tabular}{|c|c|}
        \hline
        Name & Definition \\ %& Type &
        \hline
        Empty Set & Creates a set with no elements.\\ %& $\emptyset: set[]$ &
        Set Enumeration & Literal collection of elements to create a set.\\ %& $\{x, y, ...\}: set[T]$ &
        Set Membership & The term $x \in S$ is True if $x$ can be found somewhere in $S$. \\ %& $\in: T \times set[T] \rightarrow bool$ &
        \hline
        Union & $S \cup T = \bSetT{x}{x \in S \lor x \in T}$ \\ %& $\cup: set[T] \times set[T] \rightarrow set[T]$ &
        Intersection & $S \cap T = \bSetT{x}{x \in S \land x \in T}$ \\ %& $\cap: set[T] \times set[T] \rightarrow set[T]$ &
        Difference & $S \setminus T = \bSet{x}{x \in S}{x \notin T}$ \\ %& $\setminus: set[T] \times set[T] \rightarrow set[T]$ &
        Cartesian Product & $S \times T = \bSetT{x \mapsto y}{x \in S \land y \in T}$ \\ %& $\times: set[T] \times set[V] \rightarrow relation[T,V]$ &
        \hline
        Powerset & $\mathbb{P}(S) = \bSetT{s}{s \subseteq S }$ \\ %& $\mathbb{P}: set[T] \rightarrow set[set[T]]$&
        Magnitude & $\#S = \sum_{x \in S} 1$\\ %& $\#:set[T] \rightarrow int$ &
        Subset & $S \subseteq T \equiv \forall x \in S: s \in T$ \\ %& $\subseteq: set[T] \times set[T] \rightarrow bool$ &
        Strict Subset & $S \subset T \equiv S \subseteq T \land S \neq T$ \\ %& $\subset: set[T] \times set[T] \rightarrow bool$ &
        Superset & $S \supseteq T \equiv \forall x \in T: s \in S$ \\ %& $\supseteq: set[T] \times set[T] \rightarrow bool$ &
        Strict Superset & $S \supset T \equiv S \supseteq T \land S \neq T$ \\ %& $\supset: set[T] \times set[T] \rightarrow bool$ &
        \hline
        Set Mapping & $f * S = \bSetT{f(x)}{x \in S}$\\ %& $*: (T \rightarrow T') \times set[T] \rightarrow set[T']$ &
        Set Filter & $p \triangleleft S = \bSet{x}{x \in S}{p(x)}$\\ %& $\triangleleft: (T \rightarrow bool) \times set[T] \rightarrow set[T]$ &
        % Reduction & $f / S = $ %& $/:(T\times T \rightarrow T) \times set[T] \rightarrow T$ &
        Set Quantification (Folding) & $\oplus x \cdot x \in S \mid P$\\
        Cardinality & $card(S) = \sum 1 \cdot x \in S$\\
        \hline
    \end{tabular}
    \label{tab:setOps}
\end{table}
\begin{table}[H]
    \centering
    \caption{Collection of operators on bag/multiset data types.}
    \begin{tabular}{|c|c|}
        \hline
        Name & Definition \\ %& Type &
        \hline
        Empty Set & Creates a set with no elements.\\ %& $\emptyset: set[]$ &
        Bag Enumeration & Literal collection of elements to create a set \\ & (for now, stored as a tuple of elements and number of occurrences).\\ %& $\{x, y, ...\}: set[T]$ &
        Bag Membership & The term $x \in S$ is True if $S$ contains one or more occurrences of $x$. \\ %& $\in: T \times set[T] \rightarrow bool$ &
        \hline
        Union & $S \cup T = \bag{(x, a+b)}{(x,a) \in S \land (x,b) \in T}{a, b \geq 0}$ \\ %& $\cup: set[T] \times set[T] \rightarrow set[T]$ &
        Intersection & $S \cap T = \bag{(x, min(a,b))}{(x,a) \in S \land (x,b) \in T}{a, b \geq 0}$ \\ %& $\cap: set[T] \times set[T] \rightarrow set[T]$ &
        Difference & $S - T = \bag{(x, a-b)}{(x,a) \in S \land (x,b) \in T}{a, b \geq 0 \land a-b > 0}$ \\ %& $\setminus: set[T] \times set[T] \rightarrow set[T]$ &
        % Cartesian Product & $S \times T = \bSetT{x \mapsto y}{x \in S \land y \in T}$ \\ %& $\times: set[T] \times set[V] \rightarrow relation[T,V]$ &
        % \hline
        % Powerset & $\mathbb{P}(S) = \bSetT{s}{s \subseteq S }$ \\ %& $\mathbb{P}: set[T] \rightarrow set[set[T]]$&
        % Magnitude & $\#S = \sum_{x \in S} 1$\\ %& $\#:set[T] \rightarrow int$ &
        % Subset & $S \subseteq T \equiv \forall x \in S: s \in T$ \\ %& $\subseteq: set[T] \times set[T] \rightarrow bool$ &
        % Strict Subset & $S \subset T \equiv S \subseteq T \land S \neq T$ \\ %& $\subset: set[T] \times set[T] \rightarrow bool$ &
        % Superset & $S \supseteq T \equiv \forall x \in T: s \in S$ \\ %& $\supseteq: set[T] \times set[T] \rightarrow bool$ &
        % Strict Superset & $S \supset T \equiv S \supseteq T \land S \neq T$ \\ %& $\supset: set[T] \times set[T] \rightarrow bool$ &
        % \hline
        Bag Mapping & $f * S = \bagT{(f(x), r)}{(x, r) \in S}$\\ %& $*: (T \rightarrow T') \times set[T] \rightarrow set[T']$ &
        Bag Filter & $p \triangleleft S = \bag{(x, r)}{(x, r) \in S}{p(x)}$\\ %& $\triangleleft: (T \rightarrow bool)
        Size & $size(S) = \sum r \cdot (x, r) \in S$\\
        Zero Occurrences & $(x,0) \in S \implies x \notin S$\\
        \hline
    \end{tabular}
    \label{tab:setOps}
\end{table}
\begin{table}[H]
    \centering
    \caption{Collection of operators on sequence data types.}
    \begin{tabular}{|c|c|}
        \hline
        Name & Definition \\ %& Type &
        \hline
        Empty List & Creates a list with no elements.\\ %& $[]: list[]$ &
        List Enumeration & Literal collection of elements to create a list.\\ %& $[x, y, ...]: list[T]$ &
        Construction & Alternative form of List Enumeration.\\ %& $(:):T \times list[T] \rightarrow list[T]$ &
        List Membership & The term $x \texttt{ in } S$ is True if $x$ can be found somewhere in $S$. \\ %& $\texttt{in}: T \times list[T] \rightarrow bool$ &
        \hline
        Append & $[s_1, s_2, ..., s_n] + t = [s_1, s_2, ..., s_n, t]$ \\ %& $+: list[T] \times T \rightarrow list[T]$ &
        Concatenate & $[s_1, ..., s_n] \concat [t_1, ..., t_n] = [s_1, ..., s_n, t_1, ... t_n]$ \\ %& $\concat: list[T] \times list[T] \rightarrow list[T]$ &
        Length & $\#S = \sum 1 \cdot x \texttt{ in } S$ \\ %& $\#: list[T] \rightarrow int$ &
        \hline
        List Mapping & $f * S = \ListT{f(x)}{x \texttt{ in } S}$\\ %& $*: (T \rightarrow T') \times list[T] \rightarrow list[T']$ &
        List Filter & $p \triangleleft S = \List{f(x)}{x \texttt{ in } S}{p(x)}$\\ %& $\triangleleft: (T \rightarrow bool) \times list[T] \rightarrow list[T]$ &
        Associative Reduction & $\oplus / [s_1, s_2, ..., s_n] = s_1 \oplus s_2 \oplus ... \oplus s_n$\\ %& $/:(T\times T \rightarrow T) \times list[T] \rightarrow T$ &
        Right Fold & $\texttt{foldr}(f, e, [s_1, s_2, ..., s_n]) = f(s_1 ,f(s_2 , f(..., f(s_n, e))))$\\ %& $\texttt{foldr}:(T\times V \rightarrow V) \times V \times list[T] \rightarrow V$ &
        Left Fold & $\texttt{foldl}(f, e, [s_1, s_2, ..., s_n]) = f(f(f(f(e, s_1), s_2), ...), s_n)$\\ %& $\texttt{foldl}:(T\times T \rightarrow T) \times list[T] \rightarrow T$ &
        \hline
    \end{tabular}
    \label{tab:seqOps}
\end{table}
\begin{table}[H]
    \centering
    \caption{Collection of operators on relation data types.}
    \begin{tabular}{|c|c|}
        \hline
        Name & Definition \\ %& Type &
        \hline
        Empty Relation & Creates a relation with no elements.\\ %& $\{\}:relation[]$ &
        Relation Enumeration & Literal collection of elements to create a relation.\\ %& $\{x \mapsto y, a \mapsto b,...\}: relation[T, V]$ &
        Identity & $id(S)= \bSetT{x \mapsto x}{x \in S}$\\ %& $id: set[T] \rightarrow relation[T,T]$ &
        Domain & $dom(R) = \bSetT{x}{x \mapsto y \in R}$\\ %& $dom: relation[T,V] \rightarrow set[T]$ &
        Range & $ran(R) = \bSetT{y}{x \mapsto y \in R}$\\ %& $ran: relation[T,V] \rightarrow set[V]$ &
        \hline
        Relational Image & $R[S] = \bSet{y}{x \mapsto y \in R}{x \in S}$ \\ %& $([]): relation[T,V] \times set[T] \rightarrow set[V]$ &
        Overriding & $R \ovl Q = Q \cup (dom(Q) \domsub R)$\\ %\Set{x \mapsto y}{x\mapsto y \in Q \lor (x \mapsto y \in R \land x \notin dom(Q))}$ \\ %& $\ovl: relation[T,V] \times relation[T,V] \rightarrow relation[T,V]$ &
        (Forward) Composition & $Q \circ R = \bSetT{x \mapsto z}{x \mapsto y \in R \land y \mapsto z \in Q}$\\ %& $\circ: relation[V,W] \times relation[T,V] \rightarrow relation[T,W]$ &
        Inverse & $R^{-1} = \bSetT{y \mapsto x}{x \mapsto y \in R}$ \\ %& $(^{-1}): relation[T,V] \rightarrow relation[V,T]$ &
        \hline
        Domain Restriction & $S \triangleleft R = \bSet{x \mapsto y}{x \mapsto y \in R}{x \in S}$\\ %& $\triangleleft: set[T] \times relation[T,V] \rightarrow relation[T,V]$ &
        Domain Subtraction & $S \domsub R = \bSet{x \mapsto y}{x \mapsto y \in R}{x \notin S}$\\ %& $\domsub: set[T] \times relation[T,V] \rightarrow relation[T,V]$ &
        Range Restriction & $R \triangleright S = \bSet{x \mapsto y}{x \mapsto y \in R}{y \in S}$\\ %& $\triangleright: set[V] \times relation[T,V] \rightarrow relation[T,V]$ &
        Range Subtraction & $R \ransub S = \bSet{x \mapsto y}{x \mapsto y \in R}{y \notin S}$\\ %& $\ransub: set[V] \times relation[T,V] \rightarrow relation[T,V]$ &
        % \hline % Do we need to include tests for total, surjective, injective, etc.?
        % Eventually add closures
        \hline
    \end{tabular}
    \label{tab:relOps}
\end{table}

\section{Rules}
Below is a list of rewrite rules for key abstract data types.
\subsection{Sets}

Let $S,T$ be sets, $P, E$ expressions, and $x, e$ any type.
\subsubsection{Phase 1: Set Comprehension Construction}

\noindent\begin{minipage}{\linewidth}
\begin{align}
  \tag{Predicate Operations - Union}
  S \cup T
  &\leadsto
  \bSetT{x}{x \in S \lor x \in T}
  \\
  \tag{Predicate Operations - Intersection}
  S \cup T
  &\leadsto
  \bSetT{x}{x \in S \land x \in T}
  \\
  \tag{Predicate Operations - Difference}
  S \cup T
  &\leadsto
  \bSetT{x}{x \in S \land x \notin T}
  \\
  \tag{Singleton Membership \footnote{Currently unused. We need to be careful to handle the case where $x$ is a free variable.}}
  x \in \{e\}
  &\leadsto
  x = e
  \\
  \tag{Membership Collapse \footnote{Rule only matches inside the predicate of a quantifier. Explicitly enumerating all matches for all quantuantification types and predicate cases (ANDs, ORs, etc.) would require too much boilerplate. $x$ must be bound by the encasing quantifier.}}
  x \in \oplus(E \mid P)
  &\leadsto
  P \land x = E
\end{align}
\end{minipage}

\subsubsection{Phase 2: Disjunctive Normal Form}
% https://en.wikipedia.org/wiki/Disjunctive_normal_form
\noindent\begin{minipage}{\linewidth} % Need minipage for footnote
\begin{align}
  \tag{Flatten Nested Ands}
  a \land ... \land (b \land c) \land ...
  &\leadsto
  a \land ... \land b \land c \land ...
  \\
  \tag{Flatten Nested Ors}
  a \lor ... \lor (b \lor c) \lor ...
  &\leadsto
  a \lor ... \lor b \lor c \lor ...
  \\
  \tag{Double Negation}
  \lnot \lnot x
  &\leadsto
  x
  \\
  \tag{Distribute De Morgan - Or}
  \lnot (x \lor y)
  &\leadsto
  \lnot x \land \lnot y
  \\
  \tag{Distribute De Morgan - And}
  \lnot (x \land y)
  &\leadsto
  \lnot x \lor \lnot y
  \\
  \tag{Distribute Ands}
  x \land (y \lor z)
  &\leadsto
  (x \land y) \lor (x \land z)
\end{align}
\end{minipage}

\subsubsection{Phase 3: Predicate Simplification}

\noindent\begin{minipage}{\linewidth}
\begin{align}
  % \tag{Singleton Membership One-Point Rule \footnote{From LADM 8.14. $x \in \{e\}$ must be a generator. $\oplus$ represents any quantifier, $\otimes_b$ represents any boolean operation}}
  % \oplus (E \mid x \in \{e\} \otimes_b P)
  % &\leadsto
  % E[x := e] \oplus \oplus(E \mid P)
  % \\
  \tag{Nesting \footnote{$y$ cannot occur in $P$}}
  \bSet{x,y}{P \land Q}{E}
  &\leadsto
  \bSet{x}{P}{\bSet{y}{Q}{E}}
  \\
  \tag{Generator Selection and Dummy Reassignment \footnote{The LH term must occur inside a quantifier's predicate - one match per or-clause. $P_g$ is the generator, a single clause distinguished from the rest of $\bigwedge P_i$.\\
  Dummy Reassignment uses assignment to calculate expressions outside of the if-statement. For example, $\bSet{x}{x \in S \land z = f(x)}{z}$ indirectly binds $z$ (if $z$ is free), although $z$ does not appear in the quantifier list. This may be less efficient for simple cases than directly rewriting all occurrences of z to f(x), but additional conditions that make use of z would benefit from the intermediate calculation. $P_a$ is an ordered list of such assignments (drawn from $\bigwedge P_i$), so that additional layers of indirection may be accommodated. The expectation is that LLVM will filter out superfluous assignments, though we will test this assumption later.\\Right now, we just make a naive selection of generator (ie., the first viable option). Later, this will be more intelligent.}}
  \bigwedge P_i
  &\leadsto
  (P_g, [P_a], \bigwedge_{P_i \neq P_g} P_i)
  \\
  \tag{Simplified DNF Form \footnote{The LH term must occur inside a quantifier's predicate. Combines clauses with the same generator. Requires Generator Selection to be run first}}
  (P_{g}, [P_a], \bigwedge_{P_i \neq P_g} P_i) \lor (P_{g}, [Q_a], \bigwedge_{Q_i \neq P_g} Q_i)
  &\leadsto
  (P_g, [P_a] + [Q_a],\bigwedge_{P_i \neq P_g} P_i \lor \bigwedge_{Q_i \neq P_g} Q_i)
\end{align}
\end{minipage}

\subsubsection{Phase 4: Set Code Generation}

\noindent\begin{minipage}{\linewidth}
\begin{align}
  \tag{Quantifier Generation \footnote{$\oplus$ works for any quantifier. The identity and accumulate functions are determined by the realized $\oplus$. For example, if $\oplus = \sum$, the identity is 0 and accumulate is addition.}}
  \oplus E \mid P
  &\leadsto
  \begin{minipage}[]{0.4\textwidth}
  \begin{algorithmic}
  \State $a := identity(\oplus)$
    \If{$P$} % should we use the \in operator here, or just plaintext in?
        \State $a := accumulate(a, E)$
    \EndIf
  \end{algorithmic}
  \end{minipage}
\end{align}
\end{minipage}
\noindent\begin{minipage}{\linewidth}
\begin{align}
  \tag{Disjunct conditional}
  \begin{minipage}[]{0.4\textwidth}
  \begin{algorithmic}
    \If{$\bigvee P_i$}
      \State body
    \EndIf
  \end{algorithmic}
  \end{minipage}
  &\leadsto
  \begin{minipage}[]{0.4\textwidth}
  \begin{algorithmic}
    \If{$P_0$}
      \State body
    \EndIf
    \If{$P_1$}
      \State body
    \EndIf
    \State ...
  \end{algorithmic}
  \end{minipage}
\end{align}
\end{minipage}
\noindent\begin{minipage}{\linewidth}
\begin{align}
  \tag{Conjunct conditional \footnote{Function $free$ returns clauses in $P$ that contain only free variables. $generator$ is a single clause representing the selected generator of $P$ (of form $x \in S$ where $x$ will be bound by this loop condition). $bound$ returns the clauses that contain the bound variable $x$.}}
  \begin{minipage}[]{0.4\textwidth}
  \begin{algorithmic}
    \If{($P_g, P_as, \bigwedge P_i$)}
      \State body
    \EndIf
  \end{algorithmic}
  \end{minipage}
  &\leadsto
  \begin{minipage}[]{0.4\textwidth}
  \begin{algorithmic}
    \If{$\bigwedge_{free(P_i)} P_i$}
      \For{$P_g$}
        \State $P_as$
        \If{$\bigwedge_{bound(P_i)} P_i$}
          \State body
        \EndIf
      \EndFor
    \EndIf
  \end{algorithmic}
  \end{minipage}
\end{align}
\end{minipage}
\noindent\begin{minipage}{\linewidth}
\begin{align}
  \tag{Accumulating Quantifier}
  \begin{minipage}[]{0.5\textwidth}
  \begin{algorithmic}
    \State $a := accumulator(a, \oplus(E \mid P))$
  \end{algorithmic}
  \end{minipage}
  &\leadsto
  \begin{minipage}[]{0.4\textwidth}
  \begin{algorithmic}
    \If{$P$}
      \State $a := accumulator(a, E)$
    \EndIf
  \end{algorithmic}
  \end{minipage}
\end{align}
\end{minipage}

\subsubsection{Old Rewrite System}

% Template code rule:
% \begin{minipage}[]{.45\textwidth}
% \end{minipage}
\begin{align}
  \tag{Set Construction}
  S &\rightarrow \bSetT{x}{x \in S}
  \\
  \tag{Singleton Membership}
  x \in {e} &\rightarrow x = e
  \\
  \tag{Membership Collapse}
  f(x) \cdot x \in \bSet{g(y)}{y \in S}{P} &\rightarrow f(g(y)) \cdot y \in S \mid P
  \\
  \tag{Predicate Promotion}
  x \cdot x \in S \land p(x) &\rightarrow x \cdot x \in S \mid p(x)
  \\
  \cline{1-2}
  \tag{Union}
  x \in S \cup T &\rightarrow x \in S \lor x \in T
  \\
  \tag{Intersection}
  x \in S \cap T &\rightarrow x \in S \land x \in T
  \\
  \tag{Difference}
  x \in S \setminus T &\rightarrow x \in S \land x \notin T
  \\
  \tag{Cardinality}
  card(S) &\rightarrow \sum x \in S \cdot 1
  \\
  \cline{1-2}
  \tag{Predicate Operations}
  filter(p, S) \oplus filter(q, T) &\rightarrow \bSetT{x}{(x \in S \land p(x)) \oplus_{bool} (x \in T \land q(y))}
  % \\
  % \tag{Single Operator Mapping}
  % map(f, S) \oplus T &\rightarrow \bSetT{f(x)}{x \in S \oplus_{bool} f(x) \in T}
  \\
  \tag{Filter Over Operators}
  filter(p, S) \oplus filter(p, T) &\rightarrow filter(p, S \oplus T)
  \\
  \tag{Mapping Operations}
  map(f, S) \oplus map(g, T) &\rightarrow \bSetT{z}{(x \in S \land f(x) = z) \oplus_{bool} (x \in T \land g(y) = z)}
  \\
  \tag{Mapping Over Operators}
  map(f, S) \oplus map(f, T) &\rightarrow map(f, S \oplus T)
  \\
  \tag{Map}
  map(f, S)  &\rightarrow \bSetT{f(x)}{x \in S}
  \\
  \tag{Filter}
  filter(p, S)  &\rightarrow \bSet{x}{x \in S}{p(x)}
  \\
  \cline{1-2}
  \tag{Union Subset Simplification}
  S \cup T \text{ where } S \subseteq T &\rightarrow T
  \\
  \tag{Intersection Subset Simplification}
  S \cap T \text{ where } S \subseteq T &\rightarrow S
  \\
  \tag{Difference Subset Simplification}
  S \setminus T \text{ where } S \subseteq T &\rightarrow \emptyset
  \\
  \tag{Predicate Union with Subset}
  filter(p, S) \cup filter(q, T) \text{ where } S \subseteq T &\rightarrow filter(p \lor q, T)
  \\
  \tag{Predicate Intersection with Subset}
  filter(p, S) \cap filter(q, T) \text{ where } S \subseteq T &\rightarrow filter(p \land q, S)
  \\
  \cline{1-2}
  \tag{Set generation}
  \bSet{E}{x \in S}{P}
  &\rightarrow
  \begin{minipage}[]{0.4\textwidth}
  \begin{algorithmic}
  \State $ret := \emptyset$
  \For{$x \in S$}
      \If{$P$} % should we use the \in operator here, or just plaintext in?
          ret.append(E)
      \EndIf
  \EndFor
  \end{algorithmic}
  \end{minipage}
  \\
  \tag{Summation}
  \sum x \in S \mid P \cdot E
  &\rightarrow
  \begin{minipage}[]{0.4\textwidth}
  \begin{algorithmic}
  \State $c := 0$
  \For{$x \in S$}
      \If{$P$} % should we use the \in operator here, or just plaintext in?
          $c := c + E$
      \EndIf
  \EndFor
  \end{algorithmic}
  \end{minipage}
  \\
  \tag{Composed conditional}
  \begin{minipage}[]{0.33\textwidth}
  \begin{algorithmic}
      \If{$P \land x \in S \land Q$}
          $p$
      \EndIf
  \end{algorithmic}
  where $x$ not free in $P$
  \end{minipage}
  & \rightarrow
  \begin{minipage}[]{0.27\textwidth}
  \begin{algorithmic}
  \If{$P$}
      \For{$x \in S$}
          \If{$Q$}
              $p$
          \EndIf
      \EndFor
  \EndIf
  \end{algorithmic}
  \end{minipage}
  \\
  \tag{Conjunct conditional}
  \begin{minipage}[]{0.4\textwidth}
  \begin{algorithmic}
      \If{$P \land ((x \in S \land V) \lor W) \land Q$}
          $p$
      \EndIf
  \end{algorithmic}
  where $x$ not free in $P$
  \end{minipage}
  & \rightarrow
  \begin{minipage}[]{0.4\textwidth}
  \begin{algorithmic}
  \If{$P$}
      \For{$x \in S$}
          \If{$V \land Q$}
              $p$
          \EndIf
      \EndFor
      \If{$W \land Q$}
        $p$
      \EndIf
  \EndIf
  \end{algorithmic}
  \end{minipage}
\end{align}

Additional notes and extended context:
\begin{description}
  \item[General] When the intention is clear, operands are written in a concise variable form instead of constructor notation, although the implementation should assume that variables are fully expanded by the (Set Construction) phase. The functions $map$ and $filter$ correspond to applying a function to all elements of a set and adding a predicate to the set respectively, so $map(f,S) = \bSetT{f(x)}{x \in S}$ and $filter(p, S) = \bSet{x}{x \in S}{p(x)}$.
  \item[Membership Collapse] Assumes context of the initial term results in a valid expression. For example, this rewrite would be reasonable within a quantification statement or set construction. The intention of this rule is to simplify nested set constructions and should only be used after the (Set Construction) phase.
  \item[Predicate Promotion] If a predicate within set construction cannot be considered as a loop iterator, it is promoted to a purely conditional role. The meaning of the before and after term does not necessarily change, but it clarifies the difference between candidate generators and regular predicates.
  \item[Operations involving $\oplus$] The operator $\oplus$ represents any union, intersection, or difference operation, with $\oplus_{bool}$ representing the corresponding construction operation.
  \item[Summation] The summation operation may be replaced with any reduction/folding operation, provided the correct low-level operation. More specific forms may be needed to consider non-associative operations.
  \item[Composed Conditional] $x$ not free in $P$ means $x$ must already be bound. More clearly, no candidate generator variables that have not been defined should occur in $P$. We differentiate $P$ from $Q$ so that if-statements may leave behind predicates that are not needed in the nested loop. In the case of top-level AND-separated generators that bind to the same variable, the larger generator should be promoted to a conditional.
\end{description}

\subsection{Relations}
  \begin{align}
  \tag{Image}
  R[S] &\rightarrow \bSet{x \mapsto y \in R}{x \in S}{y}
  \\
  \tag{Product}
  x \mapsto y \in S \times T &\rightarrow x \in S \land y \in T
  \\
  \tag{Inverse}
  x \mapsto y \in R^{-1} &\rightarrow y \mapsto x \in R
  \\
  \tag{Composition}
  x \mapsto y \in (Q \circ R) &\rightarrow x \mapsto z \in Q \land z' \mapsto y \in R \land z=z'
  \\
  \tag{Override}
  R \ovl Q &\rightarrow Q \cup (dom(Q) \domsub R)
  \\
  \cline{1-2}
  \tag{Domain}
  dom(R) &\rightarrow map(fst, R)
  \\
  \tag{Range}
  ran(R) &\rightarrow map(snd, R)
  \\
  \cline{1-2}
  \tag{Domain Restriction}
  S \triangleleft R &\rightarrow filter(fst \in S, R)
  \\
  \tag{Domain Subtraction}
  S \domsub R &\rightarrow filter(fst \notin S, R)
  \\
  \tag{Range Restriction}
  S \triangleright R &\rightarrow filter(snd \in S, R)
  \\
  \tag{Range Subtraction}
  S \ransub R &\rightarrow filter(snd \notin S, R)
\end{align}

\subsection{Bags}
Since bags can be interpreted as a set of tuples $(element, repetitions)$, all set operations apply, except for the overriding operations below.
\begin{align}
  \tag{Union}
  % S \cup T &\rightarrow \bag{(x, max(a,b))}{(x,a) \in S \land (x,b) \in T}{a, b \geq 0}
  S \cup T &\rightarrow \bag{(x, r)}{x \in set(S) \cup set(T)}{r = max(\#(x,S), \#(x,T))}
  \\
  \tag{Intersection}
  S \cap T &\rightarrow \bag{(x, min(a,b))}{(x,a) \in S \land (x,b) \in T}{a, b \geq 0}
  \\
  \tag{Sum}
  % S + T &\rightarrow \bag{(x, a+b)}{(x,a) \in S \land (x,b) \in T}{a, b \geq 0}
  S + T &\rightarrow \bag{(x, r)}{x \in set(S) \cup set(T)}{r = \#(x,S) + \#(x, T)}
  \\
  \tag{Difference}
  % S - T &\rightarrow \bag{(x, a-b)}{(x,a) \in S \land (x,b) \in T}{a, b \geq 0 \land a-b > 0}
  S - T &\rightarrow \bag{(x, r)}{(x,a) \in S}{r = a-\#(x, T) \land r > 0}
  \\
  \tag{Size}
  size(S) &\rightarrow \sum (x,r) \in S \cdot r
\end{align}

Additional notes and extended context:
\begin{description}
  \item[The $\#$ Operator] Defined as the number of occurrences of an element in a bag. If bags are represented by a relation, this corresponds to a direct lookup $\#(x, S) = S[x]$.
  \item[Intersection, Difference] Since the intersection and difference operators are always decreasing (ex. $S \cap T \subseteq S \land S \cap T \subseteq T$ and $S - T \subseteq S$), we can short-circuit operations that would require looping over both sets instead of just $S$. \textit{But how do we define this short-circuiting behaviour?} Intersections can make use of this property for both operands, but difference will always iterate over the first operand.
  \item[Difference] $a-b > 0 \implies a > 0$.
  \item[Sum, Union] The cast to set of $set(S)$ can be implemented by taking the domain of the bag-representing relations.
\end{description}
% TODO:
% \subsection{Lists}

\section{Implementation Representation}
Different implementations of each data type will have varying strengths and weaknesses, not only in theoretical asymptotic time and space, but in concrete real-world tests. Cache usage and additional information through object metadata may prove influential on smaller tests. Since this document is only concerned with the theoretical compiler specification, we analyze the theoretical time and space complexity, then pair gathered examples with a test plan for hardware considerations.

A first approach to tackling these type representations would likely constitute a linked list. The space requirements for enumeration are straightforward, with extra allocations for link pointers. Insertions for unordered collections or append/concat operations are $O(1)$, but $O(n)$ for indexed insertion and union with one element. Lookups for all collections are $O(n)$, but this running time is undesirable for the often-used \texttt{in} operator for set-generated code. Since linked lists naturally enforce element order, this structure may be suitable for fast-changing sequences. Although, a limited-size sequence may be better suited for a contiguous array for $O(1)$ indexing. \textit{TODO: For sequences, we should also see if trees/heaps or bloom filters could provide efficient membership checking. Bloom filters are probabilistic but can determine $\neq$ operations.}

On the other hand, hashmaps with $O(1)$ membership and element lookups are useful for all unordered collections. Relations may need bidirectional hashmaps that can efficiently handle many-to-many relations.

Compressed bitmaps may be used for sets, but require a lot of space for sparse elements.

Bags may be implemented either as a (linked) list, a set of tuples where the number of element occurrences is stored in the second tuple component, or a relation where the number of occurrences is the codomain.

\end{document}