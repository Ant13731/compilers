\documentclass[11pt]{letter}
\usepackage{latexsym}
\usepackage[empty]{fullpage}
\usepackage{titlesec}
\usepackage{marvosym}
\usepackage[usenames,dvipsnames]{color}
\usepackage{verbatim}
\usepackage{enumitem}
\usepackage[hidelinks]{hyperref}
\usepackage{fancyhdr}
\usepackage[english]{babel}
\usepackage{tabularx}
\usepackage[dvipsnames]{xcolor}
% \usepackage{xifthen}
\usepackage{fontawesome5}

% \usepackage[utf8]{inputenc}
% \usepackage{charter}

% \input{glyphtounicode}

\newcommand{\fullName}{Anthony Hunt}
\newcommand{\accentColour}{Maroon}
\newcommand{\currentCity}{Hamilton, ON}
\newcommand{\email}{\href{mailto:hunt.ant137@gmail.com}{\underline{hunt.ant137@gmail.com}}}
\newcommand{\phone}{905-870-2021}

\makeatletter
  \newcommand{\githubUrl}[1]{%
   \href{#1}{\faGithubSquare}%
  }
\makeatother
\makeatletter
  \newcommand{\linkedinUrl}[1]{%
   \href{#1}{\faLinkedin}%
  }
\makeatother

\begin{document}

\begin{center}
    \textcolor{\accentColour}{\textbf{\Huge \scshape \fullName}} \\ \vspace{1pt}
    \small \currentCity $|$ \email $|$ \phone $|$
    \linkedinUrl{https://linkedin.com/in/anthonyhunt137}
    \githubUrl{https://github.com/Ant13731}
\end{center}

\setlength\parindent{0pt}
To the OGS Awards Committee,
\setlength\parindent{24pt}

The last five years of my undergraduate degree at McMaster have afforded me incredible opportunities to better my leadership and entrepreneurial skills. Whether I was directing large group projects, guiding students through lab sessions, or working towards a lofty goal alongside like-minded people, these experiences have sparked my passion for cultivating, learning, and discussing innovative technology.

This past term, I had the pleasure of working as a teaching assistant for McMaster's third-year compiler development course, COMPSCI 3TB3. With over six years of one-on-one mathematics tutoring and intermediate-level piano teaching experience, I was excited to teach a class of 30 students in a formal academic setting. From my personal experience with this course only two years ago, I knew firsthand the impact of helpful lab sessions on the students' ability to absorb course concepts. Therefore, I strived to create intuitive lesson plans and foster open discussions, providing students with the resources to build their own understanding of complex concepts. At each lab session, I guided students through various practice problems, where they verbalized problem-solving processes and engaged with not only the “how” but the “why” of compiler development. Among the 90 students registered to my optional-attendance lab sections, over 80\% were routinely present, with several students staying late to extend discussions on various compiler topics.

In my recent co-op position as a test automation developer at Intel, I noticed significant gaps of inefficiency between the automation tools developed by my five-person team and the end-user hardware engineers. Initially, each member of my team worked independently from one another, focused on providing tools for their own specific set of clients. Becoming familiar with the testing process of all hardware engineering subgroups, I recognized the functional similarity between isolated tools and the consequential tediousness of repetitive, error-prone feature extensions. Therefore, I initiated a fundamental redesign of the entire codebase, focusing on reducing duplicate code, improving product reliability, and streamlining developer workflows.

During the initial stages of this large-scale refactoring project, many of my colleagues were understandably apprehensive of a complete codebase rework. Nevertheless, my team lead granted me the opportunity to step up and take charge of the project's development. Under his advice to gather team support by justifying the complete cost-value of the project, I decomposed all historical tooling requests into basic components and built a prototype architecture that could accommodate rapid changes in domain-related requirements. Multiple presentations and demonstrations led to positive and constructive feedback from my colleagues, with responsive feature additions and revisions attracting an increasing number of hardware engineering teams. Eventually, I guided the project and my team through the software lifecycle, improving automation tooling for over 100 engineers and expediting automation tool creation.

On a complimentary note, acting as a composer/arranger within McMaster's a cappella choir for the last five years has broadened my perspective on different kinds of leadership, particularly with respect to indirect influence over a team. As the team's most senior member, I have taken deliberate initiative to guide practices when the music directors are busy and actively encourage the growth of others, both in providing feedback and in stepping outside of their comfort zones. Further, having an empathetic understanding of the simmering anxiety among new first-year students naturally assisted my abilities to reduce nervousness and support socialization within the club. In my experience, these subtler forms of interaction can be more pivotal to a group's momentum than larger accomplishments or timelines; encouraging a growth mindset in individuals can exponentially elevate the success of a team.

Leadership can be defined as the intersection between teaching, individual contribution, and a persistent search for collaboration. All of my past leadership experiences have demonstrated that leading by teaching cannot be understated in any environment, since the act of teaching innately creates connections within a group and serves as a strong foundation of shared knowledge. Approachable leaders can then use personal connections to motivate individual development within a team and encourage peer-to-peer collaboration, ultimately uplifting aggregate productivity. However, accomplishing these tasks requires extensive preparation, informed execution, and attentive retrospection. As I begin my Master's program, I acknowledge my leadership responsibilities towards my fellow students and my learning responsibilities towards teachers and mentors, recognizing their valuable feedback and maintaining my passion for scientific exploration.

\setlength\parindent{0pt}

Thank you for your time and consideration,
\\
Anthony Hunt

\end{document}