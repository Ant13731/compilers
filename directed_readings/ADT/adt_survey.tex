\documentclass{article}

\usepackage{minted}
\usepackage{amsmath, amsthm}
\usepackage{listings}
\usepackage[svgnames]{xcolor}
\usepackage{tikz}
\usepackage{array}
\usepackage{biblatex}

\usemintedstyle{solarized-light}
\setminted{fontsize=\footnotesize, bgcolor=Beige}

% NOTE: need to run `biber adt_survey` to ensure up-to-date references
\addbibresource{adt_survey.bib}

\title{A Survey on Abstract Data Types}
\author{Anthony Hunt}

\begin{document}
\maketitle

\section{Introduction}

From the categorization of species to the periodic table of elements, the organization and manipulation
of data is integral to all aspects of science and technology. Although the collection of data is not a
new field of research by any means, the advent of computerized systems, with its ability to transfer and use
massive amounts of data in mere seconds, necessitates understandable and sensible organization structures.

Over the past 60 years, as computational resource restrictions have eased dramatically,
programming languages and software creation tools have evolved with increasingly better methods of dealing with data.
By viewing pieces of data through specific lenses, programmers can then write algorithms that operate on
any data matching the shape of those lenses, rather than the data itself. These lenses are more commonly referred to as
data types, which assign a type to a piece of data. As we will see later in this paper,
further abstractions over data types themselves enable extremely concise and readable code,
both for the programmer and computer.

The rest of this article will explore data type concepts from <X> languages, % TODO FILL IN
structures to handle errors that might appear, and a small review of ergonomics and usability of those data types.
A necessary discussion of various type systems will naturally evolve as a precursor to data types themselves.

\section{Data Types in <X>+ Languages} % TODO fill in value for X!!!!!!!!!!!!!!!!!!!!!!!!!!!!!!!!!!!!!!!

Abstract Data Types (ADTs) can look quite different in many languages, but often share the same principles and rules
that make it understandable for programmers. In its essential form, an ADT is a collection of data together with
operations that can be carried out on that data \cite{ADTspec}. The key differences between ADTs from different languages
are the conventions used to collect data and the power operations have over collected data.

In popular object-oriented programming (OOP) languages,
ADTs are approached with a ``class'' mindset, where programmers specify a blueprint for objects that
will eventually be used in a program. For better code reuse, classes can be built from pre-existing classes
through the use of inheritance, a term used to express that a new class inherits all the properties of the old class.
Objects created by classes are referred to as instances of the class and often hold
some amount of mutable data, with methods that have full access to the data. The methods can choose to make use of
``side effects'', referring to the modification of data not explicitly passed through
regular function arguments or return values. Side effects in methods have the consequence that the output of
a method may change independent of the values passed through its arguments. In other words,
the object a method has access to may contain some state that influences the outcome of the method.

At the opposite end of the programming world lies purely functional languages. ADTs in these languages
are more often referred to as data types rather than classes and are only used for collecting and operating on data.
``Objects'' created from these data types are immutable and methods are instead replaced with pure functions that
explicitly take in the relevant data type as input. The operators attached to a functionally pure ADT are functionally
pure themselves, meaning they cannot make use of any hidden state and therefore cannot have side effects.

% Type to express in all languages - a simple linked list, with append and concatenate methods
\subsection{Python}

As a dynamic interpreted scripting language with thousands of high quality libraries and a short learning curve,
Python has become one of the most dominant modern programming languages across several subdomains of computing.
The language itself is characterized mainly by its simple approach to writing readable, practical, and user-friendly code \cite{pythonZen},
used by many as a ``glue'' to easily join high performance computing libraries. The role of Python as a mediator
between optimized C libraries and human understanding, especially in the machine learning space, is adjacent to the role
of documentation in expressing the intention and meaning behind code,
rather than focusing on the small details of the code itself.
% early-age low-level languages and hand-optimized assembly. The abstractions provided from assembly to C,
% and now C to Python, are evident of the trend towards using higher level languages for a better developer experience.
% TODO refer to above comment in the rust section instead

\subsubsection{Python and Types}

The overarching data model behind Python is that everything is either an object or a relation between objects.
Almost all objects in Python are used by reference, and each object contains an identifier (memory address),
an immutable type, and then the actual value itself. In contrast to other scripting languages (namely JavaScript),
an immutable type means that Python will never implicitly recast the type of an object to fit the expected types
of functions or methods (ie. it is strongly typed).

% This behaviour is very different to types in static languages, where names are assigned a type at compile time;
% the runtime objects know their own type and determine which operations can be performed. Thankfully, because of strong
% typing,

Although Python is strongly typed, its lack of a static typing system serves as a double edged sword for
readability and reliability. With small programs, some developers may prefer to just ``get things done'',
making use of whatever combination of builtin types are available.
A mixture of sets, dictionaries, and lists with unrestricted element types are all too common in these types of scripts.
However, assigning proper data types to work with various pieces of information is essential
to the longevity and maintenance of large scale projects. Since the introduction of optional static type hints
via a library in Python 3.5 and increased support through Python 3.8,
a 2022 survey \cite{empiricalTypePython} found a steady increase in the number of GitHub repositories
making use of static type hints.

Aside from granting linters more information to statically analyze the program and warn developers of possible errors,
types also provide programmers themselves with useful information regarding their program. As reinforced
by the philosophy of TypeScript as a type-hinted version of JavaScript, data types serve as an important
channel of communication between programmers. An additional finding of the 2022 survey \cite{empiricalTypePython}
showed that typecheck errors in the linters did not prevent a large amount of code from being pushed, but were
still found to be useful outside of the linter.

Fundamentally, Python places little restriction on the programmer's actions. For example, while
data privacy within Python objects is theoretically possible through name mangled fields (beginning with double-underscores),
there is no real way to completely restrict access to an object's fields so long as that object exists in the code.
Access to deeper implementation details, like a list of current variables, list of fields in any variable, and even
interpreted bytecode is possible through the use of magic methods and variables
(denoted by two underscores at the beginning and end of the identifier).

\subsubsection{Example: A Linked List}

Since Python is primarily an OOP language, the most common form of expressing data types is through the use of classes.
A typical linked list data type in Python 3.11 would look something like:
\inputminted{python}{linked_list/main.py}

Note that Python 3.12 and later make some major changes involving scoping and the syntax of generic types \cite{pythonGenericTypeChange}.

The above \texttt{LinkedList} class expresses the recursive linking nature of the data type while demonstrating
commonly used side effects of OOP methods. Note that the return type for both \texttt{append} and \texttt{concat}
is \texttt{None}. Instead, the data contained within the \texttt{self} object is mutated in place as needed to achieve
the desired functionality.

\subsubsection{Pydantic: A data validation library}

While native Python works well for simple data types like linked lists, type hints do not provide any guarantees
that the type of value, for example, will be consistent across all elements in the list. So, while it does serve as a useful
notation tool for documenting the structure of a data type to other programmers, there are no insurance that the runtime
values for these objects are consistent with the type annotations. As with many topics in Python, a convenient library by the name
of Pydantic will enable both type hints and data validation for those types \textit{at runtime}. The Pydantic version
of our linked list will now look something like:
\inputminted{python}{linked_list/main_pydantic.py}

For this simple data type, only the init function changes. However, under the hood, Pydantic will now ensure that
the type of \texttt{value} within this linked list remains consistent across all elements, throwing an exception
otherwise. Pydantic's version of a class has the ability to transform an OOP construct into one more closely related
to functional programming. If desired, the use of decorators above field names or methods (denoted by \texttt{@decorator})
can restrict mutability, modify constructors, and even change the data validation process to ensure objects are of the correct type.

\subsubsection{Conclusion}

Python's high-level nature, excellent documentation, library support, and exposure of underlying architectures when needed
combine to create a powerful, flexible, and productive language. Although many aspects of the language are initially
hidden or abstracted for the sake of convenience, Python allows programmers to dig deeper into the implementation
when necessary and extend the language in a manner best suited for a specific task.

\subsection{TypeScript}

In JavaScript, types are very loosely bound, and many operations/functions will silently convert types
at runtime. The silent recasting of JavaScript objects combined with unintuitive defaults
can result in very confusing results for the uninitiated. For example, the expression \texttt{1 + "1"}
returns a value of 11 and the expression \texttt{[5, 11, 2].sort()} returns
\texttt{[11, 2, 5]}. In both expressions, the numerical values are converted into strings before the operation is performed.

TypeScript, as an extension of JavaScript with types, is similar to Python in the sense that both
languages started with dynamic typing and patched on static types after the fact, and both aim to resolve issues caused
by dynamic typing. TypeScript (and Python) use types not only for more assurance in program correctness,
but also for the sake of convenience. When making full use of the available typing systems, text editors
and IDEs are able to assist in new code development through intellisense and autocompletion of methods and variables.
Types of objects and documentation for those types additionally appear on mouse-hover, providing fast in-context knowledge.

The below list outlines key properties of types in TypeScript.
\begin{itemize}
    \item Types are treated as sets of values, where types can be part of multiple sets,
    a union of sets, or rarely even an intersection of sets.
    \item Types are structural. Any object containing the required fields needed to be a member of a certain type
    will be considered an object of that type, even if a different type name has been assigned to it.
    Further, if an object has additional fields not specified by the type,
    the extra fields will be ignored in the typechecking process. Essentially, as long as an object meets the minimum
    fields required by the specified type, the object is assumed to be a member of that type.
    \item TypeScript allows any literal string to act as a subset of its respective type,
    with the only allowed value being the literal itself.
    For example, a field with the type \texttt{``name''} will only accept the value of ``name''.
    However, a field with the type \texttt{String} will accept any string object.

\end{itemize}

In comparison to algebraic data types often seen in functional languages, TypeScript replaces inductive data types
with discriminated union sets. For example, our linked list type may look something like:
\inputminted{typescript}{linked_list/main.ts}

Because TypeScript allows literals to act as types, the only valid values in the ``kind'' field
of an object represented by the \texttt{LinkedList} type are our constructor names: ``nil'' or ``cons''.
Although TypeScript treats every object as a reference, similar to Python, we have not used side effects
in the \texttt{append} or \texttt{concat} functions. Instead, we explicitly accept the linked list we want to transform as
input, perform the required action, and return the head of the linked list in the output.

For a non-inductive approach to the LinkedList data type, we can use TypeScript's interfaces to create something that
looks much more like the Pydantic version:
\inputminted{typescript}{linked_list/main_interface.ts}

Finally, a pure OOP version of our LinkedList type, using classes, can be formed.
Note that the append and concat functions are now methods with side effects:
\inputminted{typescript}{linked_list/main_class.ts}





\subsection{C++}
\subsection{Rust}
\subsection{Dafny}
\subsection{Haskell}
\subsection{Agda}
\subsection{APL}
\subsection{UML}
\subsection{Event-B}
\subsection{SETL} %Maybe? Would need to cite
\subsection{Bend} %Maybe? Would need to cite
\subsection{Exploration of Conceptual Data Types}
\section{Error Handling and Undefinedness}
\section{A Word on Usability}
\section{Conclusion}


\nocite{*} % keeps all references, even those not referred to
\printbibliography %Prints bibliography

\end{document}