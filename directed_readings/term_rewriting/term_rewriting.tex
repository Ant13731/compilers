\documentclass{article}

\usepackage{minted}
\usepackage{amsmath, amsthm}
\usepackage{amssymb}
\usepackage{listings}
\usepackage[svgnames]{xcolor}
\usepackage{tikz}
\usepackage{array}
\usepackage{graphicx}
\usepackage{biblatex}

\usemintedstyle{solarized-light}
\setminted{fontsize=\footnotesize, bgcolor=Beige}

% NOTE: need to run `biber term_rewriting` to ensure up-to-date references
\addbibresource{term_rewriting.bib}

\title{A Survey on Term Rewriting for Code Optimization}
\author{Anthony Hunt}

\begin{document}
\maketitle
\tableofcontents

\section{Introduction}

% The art of formalizing text

% Since the dawn of computing, effective communication between man and machine remains...

% distilling and collecting information
% breaking down problems to build modular solutions

% In some naive, oversimplification of humanity's achievements, all knowledge

% In some broad brush stroke of humanity's achievements, all knowledge can be rendered as a transformation of one form of data to another.

% infinite monkeys with infinite typewriters

% Given an infinite number of monkeys with typewriters and an infinite amount of time, the

% sentiments are valuable especially in the field of computer science, where data reigns supreme.

% While this quote may not seem directly relevant to term writing or even compilers, it highlights an underlying

% The proverb of monkeys on a typewriter states that, given an infinite amount of time, an infinite number of monkeys will eventually produce the entire works of Shakespeare, and even all possible combinations

% As is evident by the recent boom in large language models, the collection and distillation of information has become a

% As AI has become increasingly popular

As a means of attempting to popularize the sheer vastness and absurdity of infinity with regards to information, society has often perpetuated the proverb of monkeys on a typewriter;
an infinite number of typewriter-equipped monkeys with an infinite amount of time would eventually produce the complete works of Shakespeare.
While the broad sentiment conveyed through this statement can serve as an interesting thought experiment for the masses,
computer scientists often have to deal with very real consequences of seemingly infinite swaths of data and computation on finite resources.
Indeed, with the advent of generative AI models and the internet, collecting and distilling temporarily useful information from universal entropy
has become a more pressing and arduous task than ever before.

Since computers are precise, powerful machines limited in expressivity only by their need for extremely simple instructions,
the subfield of compilers and programming languages is especially concerned with efficient, effective, and provable methods of translation.
Term rewriting is one such method that enables compilers to convert very high level language into fast, low-level executable information.

In this paper, we explore concepts and ideas surrounding the topic of simple term rewriting,
delving into the inner-workings, benefits, and limitations of such a system. Later,
we will abstract some components of the simple term rewriting for improved expressivity, control,
modularity, and overall usefulness in the context of code optimization.





\section{Term Rewriting}
\subsection{The Art of Translation}
\subsection{A (Simple) Term Rewriting System}
\subsection{Termination}
\subsubsection{Reduction Orders}
\subsubsection{Simplification Orders}
\subsection{Confluence}
\subsection{A Complete Algorithm}
\subsection{Limitations}
% \section{Code Optimization}
\section{An Improved Term Rewriting System for Code Optimization} % Could also be "A Term Rewriting System for Code Optimization"
\subsection{Rewriting Strategies}
\subsection{Code Optimization with Rewriting Strategies}
\section{Conclusion}


\pagebreak
\nocite{*} % keeps all references, even those not referred to
\printbibliography %Prints bibliography

\end{document}